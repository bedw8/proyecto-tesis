\documentclass{proyectotesis}

%%%%%%%%%%%%%%%%%%%%%%%%%%%%%%%%%%%%%%%%%%%
%%%%%%%%%%%%%%%%%%%%%%%%%%%%%%%%%%%%%%%%%%%
%%%%%%%%%%%%%%%%%%%%%%%%%%%%%%%%%%%%%%%%%%%

\title{Análisís de polarización en la Cámara de Diputados de Chile}
\author{Benjamín Matías Ortiz Edwards}
\pronom{el}
\postgrado[Magíster]{en Ciencias con mención en Física.}
%\directora{}
%\director{}
\directores{Dra. Denisse Pastén y Dr. Víctor Muñoz}
%\codirector{}
\duracion{Primer y segundo semestre 2022.}
\lugar{Grupo Planets (www.planets.cl), Departamento de Física, Facultad de Ciencias, Universidad de Chile.}
\direccion{Las Palmeras 3425, Ñuñoa, Santiago, Chile. \\Dirección Postal: Casilla 653, Santiago, Chile. Fono: (56-2) 2978 7276.}

%%%%%%%%%%%%%%%%%%%%%%%%%%%%%%%%%%%%%%%%%%%
%%%%%%%%%%%%%%%%%%%%%%%%%%%%%%%%%%%%%%%%%%%
%%%%%%%%%%%%%%%%%%%%%%%%%%%%%%%%%%%%%%%%%%%

\nacimiento{10 de febrero, 1997}
\RUN{19.359.906-0}
\telefono{(+56\;9)\:8656\;0757}
\email{bedw@bedw.me}

\begin{document}

\maketitlepage
\makepersonalinfo

%%%%%%%%%%%%%%%%%%%%%%%%%%%%%%%%%%%%%%%%%%%
%%%%%%%%%%%%%%%%%%%%%%%%%%%%%%%%%%%%%%%%%%%
\subsection{Educación}
\begin{cvlist}{}
\item[\textbf{Educación Superiror}] 
\item[\textbf{2015 - 2020}] Licenciatura en Ciencias con mención en Física, Universidad de Chile.
\item[\bf Educación Escolar]
\item[\textbf{2010 - 2014}]  Enseñanza Básica y Media, Colegio Compañia de María Seminario.
\end{cvlist}

\subsection{Experiencia Profesional}
\begin{cvlist}{}
\item[\textbf{2020}]   \textbf{Ayudante},  Métodos de la Física Matemática II, Departamento de Física, Facultad de Ciencias, Universidad de Chile, Segundo Semestre.
\item[\textbf{2021}]   \textbf{Ayudante},  Métodos de la Física Matemática II, Departamento de Física, Facultad de Ciencias, Universidad de Chile, Segundo Semestre.
\item[\textbf{2019}]  \textbf{Practicante}, "Practicas de Verano", DFC-DFI, Universidad de Chile.   
\item[\textbf{2021 - Presente}] \textbf{Programador analísta}, Centro de Inteligencia Territorial, Universidad Adolfo Ibañez. 

\end{cvlist}

\subsection{Presentaciones a Congresos}

\begin{cvlist}{}
\item[\textbf{2020}] \textbf{Ortiz Edwards, B.}, Pastén, D., y Muñoz, P. S. A complex network approach on the analysis of the Chilean presidential elections, using Twitter Data, {\it Complex Networks 2019}, Lisboa, Portugal, 10-12 de diciembre.

\end{cvlist}

\newpage

\section{Exposición General del Proyecto}
\subsection{Resumen}

\subsection{Introducción}
%La teoría de grafos busca abstraer esquemáticamente conjuntos de datos mediante criterios geométricos específicos a partir de nodos o vértices y algún tipo de relación que pueda presentar con otros nodos mediante aristas, donde estas mismas pueden presentar una orientación definida, existiendo un nodo fuente y un nodo de destino. El inicio de esta teoría fue en el año 1736 en manos de Leonhard Euler, en un artículo que trató el problema conocido como los puentes de Königsberg \cite{euler1741solutio} cuyo propósito inicial fue determinar la ruta más eficiente para cruzar todos los puentes de la ciudad, con la restricción de cruzarlos una sola vez. Euler demostró la imposibilidad del caso, sin embargo este hito ha permitido el desarrollo de importantes estudios, permitiendo desde entonces grandes avances en diversas áreas, tales como urbanismo \cite{bro2021surname},  economía \cite{gomez2020reducing}, redes biológicas \cite{ilias2020adaptive} e incluso en la selección de enfoques fundamentales en  la salud, como la propagación de enfermedades \cite{liu2020new}, entregando soluciones sustanciales a problemas complejos.\\


%Existen diferentes tipos de grafos y estructuras de representación con los que trabajar, dependiendo de las características que se buscan explorar en cada investigación. En el análisis de los datos relevantes para este trabajo, lo que interesa es modelar series de tiempo como redes complejas. Para ello, se cuenta con las herramientas de la familia de algoritmos de visibilidad, que convierten series de tiempo en grafos donde la estructura de la serie se conserva en la topología del grafo \cite{lacasa2008time}, logrando construir un puente natural entre la teoría de redes complejas y el análisis de series de tiempo. %En este grafo cada nodo corresponde, en el mismo orden, a datos en serie, y se conectan dos nodos si existe visibilidad (oblicua) entre los datos correspondientes, es decir, si existe una línea recta que conecta los datos en serie, siempre que que esta ``línea de visibilidad" no cruce ninguna altura de datos intermedia.\\ %\cite{lacasa2012time}

Las redes complejas son muy útiles para caracterizar sistemas con muchos elementos que interactuan entre si. Estas nos permiten modelar distintos sistemas mediante una representación de grafos, con nodos y conexiones, para luego obtener distintas propiedades del sistema a partir de métricas, o propiedades estadísticas o topológicas de la red.\\

Este proyecto busca estudiar la polarización de la Cámara de Diputados de Chile, a partir de las interacciones entre los miembros de esta, haciendo uso de redes. En particular, se pretende inferir propiedades de la Cámara a partir de la coordinación (o descoordinación) entre parlamentarios al momento de votar en Sala. \\

Para estudiar la polarización de una red es necesario \textbf{detectar comunidades} en ella, a partir de las conexiones y sus pesos (intensidades). Luego, una red se considera polarizada cuando los nodos de una misma comunidad tienen interacciones más fuertes entre sí, mientras que más debiles (o nulas) con los nodos de distintas comunidades. Sin embargo, Neal propone un segundo tipo de polarización, lo que denomina ``polarización fuerte'', la cual consiste en que adem


Actualmente existe una gran cantidad de trabajos con un enfoque similar, en que se estudia la polarización de parlamentos de distitos paises usando analisis de redes, considerando las relaciones entre parlamentarios. En estos los \textbf{parlamentarios se representan como nodos, y las relaciones entre ellos como conexiones}. Algunos autores definen las interacciones (conexiones) a partir de la coordinacion (o descordinación) al momento de votar, mientras que otros a partir del patrocinio en conjunto a los propuestas de ley. 

Para detectar comunidades hay diversos algoritmos, pero sin duda los mas utilizados son aquellos que maximizan la Modularidad ($Q$), definida por Newmany Garvin. Sin embargo esta, solo válida cuando las interacciónes 


Uno de ellos es el algoritmo ``spin-glass'' de Reicherdat, el cual consiste el la minimización de un Hamiltoniano, demostrando ser equivalente a la maximización de la Modularidad.




\subsection{Objetivos}

En el contexto del estudio de interacciones entre los miembros de la Cámara de Diputados de Chile, se busca evaluar los algoritmos de detección de comunidades basados en la maximización de modularidad como una medida ``válidad'' de polarización política. Para ello se propone comparar los valores de modularidad y las particiones de comunidades obtenidas del algoritmo ``spin-glass'' de Reicherdat, con el índice del triángulo, al considerar interacciones negativas (repulsivas). 

Posteriormente se buscará 

\subsubsection*{Objetivos Específicos}
\begin{itemize}
\item    {\bf Análisis de Redes Complejas.} Caracterizar la interacción entre miembros de la Cámara de Diputados de Chile, por medio de la representación del sistema como red compleja, a partir de los datos públicos de las votaciones en sala, u otros similares.

\item{\bf Comparar modularidad con índice del triángulo.} Relacionar los modelos sociofisicos exitentes, 

\item {\bf Elaboración de nuevo hamiltoniano.} 

\end{itemize}

\subsection{Metodología}
\subsubsection{Análisis de Redes Complejas.}

Los votos de una votación en Sala $j$ se representará como un vector 
\begin{equation}
    v^j = \{u_1,\dots,u_n\}
\end{equation}
con $u_i$ la opción que el o la parlamentaria $i$ votó. Esta será 1 si el voto fue a favor, -1 si fue en contra y 0 en caso de abstención o ausencia.

Realizando un producto externo entre estos vectores, se obtiene la matriz $M^j$ que representa la coordinación de los parlamentarios en la votación $j$. De esta forma $M^j_{kl}$ será 1 si los parlamentarios $k$ y $l$ votaron lo mismo, -1 si votaron opciones opuestas, y 0 si alguno se abstuvo o estuvo ausente. Entoces, ahora, para $n$ votaciones, tendremos
\begin{equation}
    A_{kl} = \sum_j M^j_{kl}
\end{equation}

%Se busca caracterizar la interacción entre miembros de la Cámara de Diputados de Chile, por medio de la representación del sistema como red compleja, a partir de los datos públicos de las votaciones en sala. 
%Se propone usar el siguiente modelo para la construcción de la red.\\

%Se considerará cada diputado/a un nodo, y estos se conectarán entre sí, si coinciden al votar en las distintas votaciones en Sala. Por ejemplo si el parlamentario $a$, en una votación en particular, vota lo mismo (A favor, En contra, Abstención) que el parlamentario $b$, entonces hay una conexión entre $a$ y $b$ 
%
%
%Para ello se usarán algoritmos de detección de comunidades basados en principios de la mecánica estadistica, tienendo así un punto de encuentro con el modelo antes señalado.   

\subsubsection{Modelo de spin-glass.} Serge Galam propone el siguiente hamiltoniano para el sistema de alianzas o fragmentación entre paises.
\begin{equation}
    H = \frac{1}{2} \sum_{i>j}^n \{G_{i,j} + \epsilon_i \epsilon_j\}\eta_i \eta_i - \sum_i^n \beta_i b_i \eta_i
,\end{equation}
con $\{\eta_i = \pm 1 \}$ las variables de Ising, que indican la elección del país $i$ de pertenecer a la alianza B $(\eta_i = 1)$ o a la alianza B $(\eta_i = -1)$. Por otro lado $\epsilon_i$ indica la alianza a la cual el pais $i$ \textit{deberia} pertenecer, principalmente por motivos historicos

\subsubsection{Vinculación red-modelo y re-caracterizción.} Relacionar las conexiones de las redes elaboradas con las interacciones del modelo. Luego, obtener cantidades como la energía, temperatura o magnetización, propias de la física estadística, para este sistema de estudio, permitiendo así, re-caracterizarlo. Por otro lado, también dar una re-interpretación al concepto de ``polarización política''  basada en los algoritmos de detección de comunidades mencionados en el punto anterior. 
\subsection{Trabajo adelantado}

\subsection{Sugerencia de plan de trabajo}
\begin{itemize}
\item \textbf{Semestre Otoño, 2022.} Este semestre será dedicado para el estudio de los modelos sociofíscos como para la elaboración del nuevo modelo, y las simulaciones computacionales correspondientes. También retomar el análisis de redes complejas, considerando el trabajo ya realizado.

\item \textbf{Semestre Primavera, 2022.} En este semestre se estudiará el vinculo del modelo sociofísico con el análisis de redes. Se desarrollaran nuevos cálculos computacionales propios de la re-caracterización del sistema. En este periodo se contempla la redacción de tesis. En caso de ser necesario, esto último pude desarrollarse en parte del siguiente semestre (Otoño 2023). 
\end{itemize}

\subsection{Materiales y financiamiento}

Para llevar a cabo este proyecto se necesitan utilizar las dependencias del Departamento de Física en la Facultad de Ciencias de la Universidad de Chile. % Principalmente, el trabajo se desarrollará en la oficina y los computadores del grupo de investigación PLANETS con acceso a internet.

\subsubsection*{Recursos materiales}

\begin{itemize}
\item Puesto de trabajo y materiales en oficina
%\item Acceso al Cluster en el departamento de Física de la universidad de Chile, para realizar cálculos numéricos y análisis de datos.

\nocite{*}
\end{itemize}
\subsubsection*{Financiamiento}
\begin{itemize}
\item
\end{itemize}

%\bibliographystyle{abbrv}
%\bibliography{citas.bib}

\end{document}


 
 
 
 
 
 
