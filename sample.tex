\documentclass{proyectotesis}
%\documentclass[hyperref]{proyectotesis} % para mostrar hipervinculos

% Manejo de referencias/bibliografía con el paquete BibLaTex
%\addbibresource{refs.bib} % descomentar para cargar referencias
%%%%%%

\title{El pulento título de proyecto}
\author{Guaripolo, el personaje favorito}
\pronom{le}
\postgrado[Magíster o Doctorado]{en algo.}
%\anadirdireccion[Directora]{}
%\anadirdireccion[Director]{}
\anadirdireccion[Directores]{Dra. Patana y Dr. Juan Carlos Bodoque}
%\anadirdireccion[Co\,-Director]{}
\duracion{Tal y tal semestre.}
\lugar{Grupo tanto, Departamento X, Facultad Y, Universidad de Chile.}
\direccion{221B Baker Street St, Londres, Inglaterra}

%%%%%%%%%%%%%%%%%%%%%%%%%%%%%%%%%%%%%%%%%%%
%%%%%%%%%%%%%%%%%%%%%%%%%%%%%%%%%%%%%%%%%%%
%%%%%%%%%%%%%%%%%%%%%%%%%%%%%%%%%%%%%%%%%%%

\nacimiento{31 de febrero, 2050}
\RUN{12.345.678-k}
\telefono{(+56\;9)\:9999\;9999}
\email{correo@gmail.com}

\begin{document}
%%%%%%%%%%%%%%%%%%%%%%%%%%%%%%%%%%%%%%%%%%%
\setlength{\cvlabelwidth}{45mm}             % Espaciado del entorno cvlist
%%%%%%%%%%%%%%%%%%%%%%%%%%%%%%%%%%%%%%%%%%%

\maketitlepage
\makepersonalinfo

%%%%%%%%%%%%%%%%%%%%%%%%%%%%%%%%%%%%%%%%%%%
%%%%%%%%%%%%%%%%%%%%%%%%%%%%%%%%%%%%%%%%%%%
\subsection{Educación}
\begin{cvlist}{}
\item[\bf Educación Escolar]
\item[\textbf{2000 - 2001}]  Enseñanza Básica y Media, Liceo/Colegio Titirilquén.
\item[\textbf{Educación Superiror}] 
\item[\textbf{2000 - 2001}] Alguna carrera, Instituto Aplaplac.
\end{cvlist}

\subsection{Experiencia Profesional}
\begin{cvlist}{}
\item[\textbf{Fecha}]  \textbf{Cargo}, Lugar.   
\item[\textbf{Primavera 2000}]   \textbf{Ayudante},  Curso, Departamento X, Facultad Y, Universidad de Chile.
\item[\textbf{Primavera 2000}]   \textbf{Ayudante},  Curso, Departamento X, Facultad Y, Universidad de Chile.

\end{cvlist}

\subsection{Presentaciones en Congresos}

\begin{cvlist}{}
\item[\textbf{2000}] \textbf{Bodoque, J.C}, Triviño, T. Título, {\it Congreso}, Lugar, 31 de febrero.
\item[\textbf{2000}] \textbf{Bodoque, J.C}, Triviño, T. Título, {\it Congreso}, Lugar, 31 de febrero.

\end{cvlist}

\subsection{Publicaciones}

\begin{cvlist}{}
\item[\textbf{2000}] \textbf{Bodoque, J.C}, Triviño, T. Título, {\it Revista}, Vol 1, p300, DOI XXXXXXXXXXXXX.
\item[\textbf{2000}] \textbf{Bodoque, J.C}, Triviño, T. Título, {\it Revista}, Vol 1, p300, DOI XXXXXXXXXXXXX.

\end{cvlist}

\newpage

\section{Exposición General del Proyecto}
\subsection{Resumen}

Resumen

\subsection{Introducción}

\subsection{Objetivos}

\subsubsection*{Objetivos Específicos}
\begin{itemize}
\item    {\bf Objetivo 1} Breve descripción objetivo 1
\item    {\bf Objetivo 2} Breve descripción objetivo 2
\item    {\bf Objetivo 3} Breve descripción objetivo 3

\end{itemize}

\subsection{Metodología}
\subsubsection{Desarrollo objetivo 1.}
\subsubsection{Desarrollo objetivo 2.} 
\subsubsection{Desarrollo objetivo 3.} 

\subsection{Trabajo adelantado}

\subsection{Sugerencia de plan de trabajo}
\begin{itemize}
\item \textbf{Semestre Otoño, 2000.} En este semestre haremos tangananica.

\item \textbf{Semestre Primavera, 2022.} En este semestre haremos tangananá. En este periodo se contempla la redacción de tesis. En caso de ser necesario, esto último podrá desarrollarse en parte del siguiente semestre. 
\end{itemize}

\subsection{Materiales y financiamiento}

Para llevar a cabo este proyecto se necesita un reactor nuclear. 

\subsubsection*{Recursos materiales}

\begin{itemize}
\item Puesto de trabajo y materiales en oficina
\end{itemize}

\subsubsection*{Financiamiento}
\begin{itemize}
\item Proyecto FONDECYT N$^\circ$ 1234567.
\end{itemize}

%% \printbibliography[heading=bibintoc] %% descomentar para mostrar referencias

\end{document}






